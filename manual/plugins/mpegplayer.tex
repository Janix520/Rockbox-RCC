% $Id$ %
\subsection{MPEG Player}
The Mpeg Player is a video player plugin capable of playing back MPEG-1 and 
MPEG-2 video streams with MPEG audio multiplexed into \fname{.mpg} files.

To play a video file, you just select it in the Rockbox \setting{File Browser}.
If your file does not have the \fname{.mpg} extension but is encoded in the
supported format, you will need to use the \setting{Open With...} context menu
option and choose \setting{mpegplayer}.

\begin{btnmap}
    \opt{GIGABEAT_S_PAD}{\ButtonSelect{} or \ButtonPlay}
    \opt{GIGABEAT_PAD}{\ButtonSelect{} or \ButtonA}
    \nopt{GIGABEAT_S_PAD,GIGABEAT_PAD}{\ActionWpsPlay}
       \opt{HAVEREMOTEKEYMAP}{& } 
    & Pause / Resume\\
    \ActionWpsStop
       \opt{HAVEREMOTEKEYMAP}{& }
    & Stop\\
    \nopt{GIGABEAT_S_PAD,GIGABEAT_PAD}{\ActionWpsVolUp{} / \ActionWpsVolDown}
    \opt{GIGABEAT_S_PAD,GIGABEAT_PAD}{\ButtonLeft{} or  \ButtonVolDown{} /
        \ButtonRight{} or \ButtonVolUp}
       \opt{HAVEREMOTEKEYMAP}{& }
    & Adjust volume up / down\\
    \nopt{GIGABEAT_S_PAD,GIGABEAT_PAD}{\ActionWpsSkipPrev{} / \ActionWpsSkipNext}
    \opt{GIGABEAT_S_PAD,GIGABEAT_PAD}{\ButtonUp{} / \ButtonDown}
       \opt{HAVEREMOTEKEYMAP}{& }
    & Rewind / Fast Forward\\
    \opt{IRIVER_H100_PAD,IRIVER_H300_PAD}{\ButtonMode}
    \opt{IPOD_4G_PAD,IPOD_3G_PAD,GIGABEAT_PAD,GIGABEAT_S_PAD,MROBE100_PAD,PBELL_VIBE500_PAD}
        {\ButtonMenu}
    \opt{IAUDIO_X5_PAD}{\ButtonRec}
    \opt{IRIVER_H10_PAD}{\ButtonRew}
    \opt{SAMSUNG_YH92X_PAD,SAMSUNG_YH820_PAD}{\ButtonLeft}
    \opt{SANSA_E200_PAD,SANSA_FUZE_PAD,SANSA_C200_PAD,SANSA_CLIP_PAD}{\ButtonSelect}
       \opt{HAVEREMOTEKEYMAP}{& }
    & Open the MPEG Player menu\\
\end{btnmap}

When a video file is selected, the Start Menu will be displayed, unless it is 
disabled via the \setting{Resume Options} (see below). In the latter case the video 
will start playing immediately.

Start Menu

\begin{description}
\item[Play from beginning] Resume information is discarded and the video plays
    from the start.
\item[Resume at: mm:ss] Resume video playback at stored resume time mm:ss
    (start of the video if no resume time is found).
\item[Set start time] A preview screen is presented consisting of a
    thumbnail preview and a progress bar where the user can select a start time
    by `seeking' through the video. The video playback is started by pressing
    the select button.
\item[Settings] Open \setting{Settings} submenu -- see below.
\item[Quit mpegplayer] Exit the plugin.
\end{description}

Main Menu

\begin{description}
\item[Settings] Open \setting{Settings} submenu -- see below.
\item[Resume playback] Return to playback screen.
\item[Quit mpegplayer] Exit the plugin.
\end{description}

Settings Menu

\begin{description}
\item[Display Options] Open \setting{Display Options} submenu -- see below.
\item[Audio Options] Open \setting{Audio Options} submenu -- see below.
\item[Resume Options] (default: Start menu) Enable/disable the start menu.
\item[Play Mode] (default: Single) Set to \setting{All} to play multiple
    \fname{.mpg} files in the directory continuously.
\item[Clear all resumes: x] Discard all x resume points.
\end{description}

Display Options Menu

\begin{description}
\item[Dithering] (default: off) Prevent banding effects in gradients by blending
    of colours. (only available on Sansa e200, Sansa c200 and Gigabeat F/X)
\item[Display FPS] (default: off) This option displays (once a second - if your
    video is full-screen this means it will get overwritten by the video and
    appear to flash once per second) the average number of frames decoded per
    second, the total number of frames skipped (see the Skip Frames option),
    the current time (in 100~Hz ticks) and the time the current frame is due to
    be displayed.
\item[Limit FPS] (default: on) With this option disabled, mpegplayer will
    display the video as fast as it can. Useful for benchmarking.
\item[Skip frames] (default: on) This option causes mpegplayer to attempt to
    maintain realtime playback by skipping the display of frames - but these
    frames are still decoded. Disabling this option can cause loss of A/V sync.
\opt{backlight_brightness}{
  \item[Backlight Brightness] (default: Use setting) Choose brightness to use
    during video playback. Set to \setting{Use setting} to use the Brightness
    setting.
}
\end{description}

Audio Options Menu

\begin{description}
\item[Tone Controls] (default: force off) Use the bass and treble control
    settings or force them off.
\item[Channel Modes] (default: force off) Use the channel configuration setting
    or force Stereo mode.
\item[Crossfeed] (default: force off) Use the Crossfeed setting or force
    crossfeed off.
\item[Equalizer] (default: force off) Use the Equalizer setting or force the
    equalizer off.
\item[Dithering] (default: force off) Use the Dithering setting or force
    audio dithering off.
\end{description}

See this page in the Rockbox wiki for information on how to encode your videos
to the supported format. \wikilink{PluginMpegplayer}
