\subsection{PictureFlow}
\screenshot{plugins/images/ss-pictureflow}{PictureFlow}{img:pictureflow}
PictureFlow provides a visualisation of your albums with their associated cover
art. \opt{swcodec}{It is possible to start playback of the selected
album from PictureFlow. Playback will start from the selected track. The 
PictureFlow plugin will continue to run while your tracks are played.}

\opt{hwcodec}{
\note{PictureFlow is a visualisation only. It cannot be used to select and
play music.  Also, using this plugin will cause playback to stop.}
}

\subsubsection{Requirements}
PictureFlow uses both the album art (see \reference{ref:album_art}) and 
database (see \reference{ref:database}) features of Rockbox.
It is therefore important that these are working correctly before attempting
to use PictureFlow. In addition, there are some other points of which to be
aware:

  \begin{itemize}
    \item PictureFlow will accept album art larger than the dimensions of the
    screen, but the larger the dimensions, the longer they will take to scale.
  \end{itemize}

\subsubsection{Keys}
    \begin{btnmap}
        \opt{scrollwheel,IRIVER_H10_PAD,PBELL_VIBE500_PAD,MPIO_HD300_PAD,SAMSUNG_YH92X_PAD%
            ,SAMSUNG_YH820_PAD}{
            \ActionStdPrev{} / \ActionStdNext
                &
            \opt{HAVEREMOTEKEYMAP}{
                &}
            Scroll through albums / track list
                \\
        }
        
        % only scroll wheel and `strip' targets use the same action in album and track list
        \nopt{scrollwheel,IRIVER_H10_PAD,PBELL_VIBE500_PAD,MPIO_HD300_PAD,SAMSUNG_YH92X_PAD%
             ,SAMSUNG_YH820_PAD}{%
            % currently the M3 does not use buttons of the main unit which has no display
            \nopt{IAUDIO_M3_PAD,MPIO_HD200_PAD,touchscreen}{\ButtonLeft{} / \ButtonRight}
            \opt{MPIO_HD200_PAD}{FIXME}
            \opt{touchscreen}{\TouchMidLeft{} / \TouchMidRight}
                &
            \opt{HAVEREMOTEKEYMAP}{
                \opt{IAUDIO_M3_PAD,GIGABEAT_RC_PAD}{\ActionRCStdPrev{} / \ActionRCStdNext}
                &}
            Scroll through albums
                \\

            \nopt{IAUDIO_M3_PAD}{\ActionStdPrev{} / \ActionStdNext}
                &
            \opt{HAVEREMOTEKEYMAP}{
                % even though the M3 uses an Iaudio remote, mapping differs when used with M/X5
                \opt{IAUDIO_M3_PAD}{\ButtonRCLeft{} / \ButtonRCRight}
                \opt{GIGABEAT_RC_PAD}{\ButtonRCVolUp{} / \ButtonRCVolDown}
                &}
            Scroll through track list
                \\
        }

        \nopt{IAUDIO_M3_PAD}{%
            \nopt{ONDIO_PAD,IRIVER_H10_PAD,RECORDER_PAD,touchscreen,PBELL_VIBE500_PAD%
                 ,SANSA_FUZE_PAD,MPIO_HD200_PAD,MPIO_HD300_PAD,SAMSUNG_YH92X_PAD%
                 ,SAMSUNG_YH820_PAD}
                 {\ButtonSelect}
            \opt{ONDIO_PAD}{\ButtonUp} 
            \opt{IRIVER_H10_PAD,PBELL_VIBE500_PAD,SAMSUNG_YH92X_PAD,SAMSUNG_YH820_PAD}{\ButtonRight} 
            \opt{RECORDER_PAD}{\ButtonOn} 
            \opt{touchscreen}{\TouchCenter}
            \opt{SANSA_FUZE_PAD}{\ButtonRight}
            \opt{MPIO_HD200_PAD}{\ButtonFunc}
            \opt{MPIO_HD300_PAD}{\ButtonEnter}
        }
            &
        \opt{HAVEREMOTEKEYMAP}{
            \opt{IAUDIO_M3_PAD}{\ButtonRCPlay}
            \opt{GIGABEAT_RC_PAD}{\ButtonRCFF}
            &}
        Enter track list
            \nopt{ONDIO_PAD}{/ Play album from selected track}
            \\
            
        % Ondio uses a different button in album list and track list
        \opt{ONDIO_PAD}{
            \ButtonMenu
                &
            Play album from selected track in track list
                \\
        }
        
        \nopt{IAUDIO_M3_PAD,MPIO_HD200_PAD,MPIO_HD300_PAD,touchscreen,SANSA_FUZEPLUS_PAD}{\ButtonLeft}
        \opt{MPIO_HD200_PAD}{\ButtonRec}
        \opt{MPIO_HD300_PAD}{\ButtonMenu}
        \opt{SANSA_FUZEPLUS_PAD}{\ButtonLeft{} or \ButtonBack}
        \opt{touchscreen}{
            \opt{COWON_D2_PAD}{\ButtonPower{} or}
            \TouchBottomRight}
            &
        \opt{HAVEREMOTEKEYMAP}{
            \opt{IAUDIO_M3_PAD,GIGABEAT_RC_PAD}{\ActionRCStdCancel}
            &}
        Exit track list
            \\

        \nopt{IAUDIO_M3_PAD,SANSA_FUZEPLUS_PAD}{\ActionStdMenu}
            \opt{SANSA_FUZEPLUS_PAD}{long \ButtonSelect}
            &
        \opt{HAVEREMOTEKEYMAP}{
            \opt{IAUDIO_M3_PAD,GIGABEAT_RC_PAD}{\ActionRCStdMenu}
            &}
        Enter menu
            \\

        \nopt{IAUDIO_M3_PAD}{%
            \opt{IRIVER_H100_PAD,IRIVER_H300_PAD,RECORDER_PAD,ONDIO_PAD}{\ButtonOff}
            \opt{IAUDIO_X5_PAD,GIGABEAT_PAD,GIGABEAT_S_PAD,SANSA_E200_PAD,SANSA_CLIP_PAD%
                ,MROBE100_PAD,SANSA_FUZEPLUS_PAD}{\ButtonPower}
            \opt{SANSA_C200_PAD,IRIVER_H10_PAD}{Long \ButtonPower}
            \opt{IPOD_4G_PAD,IPOD_3G_PAD}{Long \ButtonMenu}
            \opt{SANSA_FUZE_PAD}{Long \ButtonHome} 
            \opt{PBELL_VIBE500_PAD,SAMSUNG_YH92X_PAD,SAMSUNG_YH820_PAD}{\ButtonRec}
            \opt{MPIO_HD200_PAD}{FIXME}
            \opt{MPIO_HD300_PAD}{Long \ButtonMenu}
            \opt{touchscreen}{
                \opt{COWON_D2_PAD}{Long \ButtonPower{} or}
                \TouchBottomRight{} (in album view)}
        }
            &
        \opt{HAVEREMOTEKEYMAP}{
            \opt{IAUDIO_M3_PAD}{\ButtonRCRec}
            \opt{GIGABEAT_RC_PAD}{\ButtonRCRew}
            &}
        Exit PictureFlow
            \\
            
    \end{btnmap}

\subsubsection{Main Menu}
\begin{description}
  \item[Go to WPS.] Leave PictureFlow and enter the while playing screen.
  \opt{swcodec}{\item[Playback Control.] Control music playback from within the plugin.}
  \item[Settings.] Enter the settings menu.
  \item[Return.] Exit menu.
  \item[Quit.] Exit PictureFlow plugin.
\end{description}

\subsubsection{Settings Menu}

\begin{description}
  \item[Show FPS.] Displays frames per second on screen.
  \item[Spacing.] The distance between the front edges of the side slides, i.e. changes
  the degree of overlap of the side slides. A larger number means less overlap. Scales with zoom.
  \item[Centre margin.] The distance, in screen pixels, with zoom at 100, between
  the centre and side slides. Scales with zoom.
  \item[Number of slides.] Sets the number of slides at each side, including the
  centre slide. Therefore if set to 4, there will be 3 slides on the left,
  the centre slide, and then 3 slides on the right.
  \item[Zoom.] Changes the distance at which slides are rendered from the ``camera''.
  \item[Show album title.] Allows setting the album title to be shown above or
  below the cover art, or not at all.
  \item[Resize Covers.] Set whether to automatically resize the covers or to leave
  them at their original size.
  \item[Rebuild cache.] Rebuild the PictureFlow cache. This is needed in order
  for PictureFlow to pick up new albums, and may occasionally be needed if albums
  are removed.
\end{description}
